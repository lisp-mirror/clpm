\documentclass[format=sigconf]{acmart}

\usepackage[utf8]{inputenc}
\usepackage{hyperref}
\usepackage{listings}

\title{Common Lisp Project Manager}
\author{Eric Timmons}
\email{etimmons@mit.edu}

\affiliation{
  \institution{CSAIL, Massachusetts Institute of Technology}
  \streetaddress{32 Vassar St}
  \city{Cambridge, MA}
  \country{USA}
}

\acmConference[ELS’21]{the 14th European Lisp Symposium}{May 03--04 2021}{Online, Everywhere}
\acmISBN{}
\acmDOI{10.5281/zenodo.4716466}
\setcopyright{rightsretained}
\copyrightyear{2021}

\begin{CCSXML}
<ccs2012>
<concept>
<concept_id>10011007.10011006.10011071</concept_id>
<concept_desc>Software and its engineering~Software configuration management and version control systems</concept_desc>
<concept_significance>500</concept_significance>
</concept>
<concept>
<concept_id>10011007.10011006.10011072</concept_id>
<concept_desc>Software and its engineering~Software libraries and repositories</concept_desc>
<concept_significance>300</concept_significance>
</concept>
<concept>
<concept_id>10011007.10011006.10011066.10011070</concept_id>
<concept_desc>Software and its engineering~Application specific development environments</concept_desc>
<concept_significance>300</concept_significance>
</concept>
</ccs2012>
\end{CCSXML}

\ccsdesc[500]{Software and its engineering~Software configuration management and version control systems}
\ccsdesc[300]{Software and its engineering~Software libraries and repositories}
\ccsdesc[300]{Software and its engineering~Application specific development environments}


\begin{document}

\begin{abstract}
  In this paper we describe and demonstrate the Common Lisp Project Manager
  (CLPM), a new addition to the Common Lisp dependency management
  ecosystem. CLPM provides a superset of features provided by the Quicklisp
  client, the current de facto project manager, while maintaining compatibility
  with both ASDF and the primary Quicklisp project distribution. These new
  features bring the Common Lisp dependency management experience closer to
  what can be achieved in other programming languages without mandating a
  particular development workflow/environment and without sacrificing Common
  Lisp's trademark interactive style of development.
\end{abstract}

\maketitle

\section{Introduction}

One way to manage the complexity of software is to reuse code by packaging up a
project and using it as a dependency in another. Unfortunately, this ends up
introducing another level of complexity: you have to manage your dependencies
and their versions as well! In the Common Lisp world, this management is
usually performed by three interacting pieces: Another System Definition
Facility (ASDF) \cite{asdf} to ensure all dependencies are compiled and loaded
in the right order and the version of each is sufficient, the Quicklisp client
\cite{ql} for installing the dependencies locally and configuring ASDF to find
them, and a Quicklisp distribution containing an index that lists available
projects and metadata about them.

While effective and widely used, this combination of components is missing many
features that are taken for granted in other programming language specific
package management ecosystems. In this paper we describe the Common Lisp
Project Manager (CLPM)\footnote{CLPM is in the process of being renamed from
  the Common Lisp Package Manager. While in most communities ``package''
  typically refers to some installable unit of software, it unfortunately
  collides with the use of ``package'' to describe symbol namespaces in Common
  Lisp}, a potential replacement for the Quicklisp client in the Common Lisp
dependency management ecosystem that provides many of these missing features.

We begin by first defining some terminology used throughout the paper. We then
provide an overview of the tasks a dependency management solution must
perform. Next we highlight several of the key features that CLPM implements,
along with the overarching design philosophy. We then describe the high-level
design of CLPM. Last, we provide information on how to obtain CLPM and several
examples on how to use it.

\section{Terminology}

In this section, we describe some of the terminology used throughout the rest
of the paper. While the broad ideas are general, we do focus on Common Lisp and
particularly on projects that use ASDF.

\begin{description}
\item[Project] Primary development unit of code. Typically corresponds to the
  contents of a single version control system (VCS) repository, such as
  git.
\item[Release] A snapshot of a project. Typically identified using a version
  number, date, or VCS commit identifier. May contain multiple systems.
\item[System] Defined via ASDF's \verb=defsystem=. Describes components
  (typically source code files) to be compiled/loaded, the version of the
  system, and dependencies on other systems.
\item[Dependency] A tuple $\langle f, s, v \rangle$. $f$ is a feature
  expression, $s$ is a system name, and $v$ is a version constraint. States
  that if $f$ is satisfied when evaluated against \verb=*features*=, then a
  system with the name $s$ is required whose version satisfies
  $v$.\footnote{Currently, ASDF only supports version constraints that specify
    a minimum version number.}
\item[Source/Index] A collection of project and system metadata.
\end{description}

\section{Dependency Management Overview}

Broadly speaking, there are three tasks that any dependency management solution
must perform: dependencies must be resolved, installed, and built. In this
section, we explain each task, describe the Common Lisp tools involved in each
(when using the current de facto workflow), and then provide further grounding
by briefly explaining what tools are used for each task in selected other
language ecosystems.

In the building phase, every dependency must be compiled and/or loaded, in the
correct order. Additionally, there should be some feedback if dependencies are
not met (for instance, system A needs version 2 of system B, but only version 1
is available). In current practice, ASDF performs this function. It uses system
definitions to determine what dependencies any given system. Then it finds
every dependent system, checks the version constraints, produces a plan that
contains an ordered set of operations to perform, and executes it.

However, ASDF has no support for installing dependencies; they must be placed
on the file system or in the Lisp image by another tool. While it could be done
manually, this installation phase is routinely performed by the Quicklisp
client. The client fetches tarballs from the internet, unpacks them to the
local file system, and configures ASDF so that it can find them.

Deciding which releases need to be installed is called dependency resolution. A
highly manual approach to dependency resolution is to install a single release,
try building it, see what's missing, install another release, and
repeat. However, the process can be sped up using an index that specifies a
list of known systems and their direct dependencies. The index can then be
consulted recursively for each direct dependency (and their dependencies, and
their dependencies, and so on) to produce a complete list of releases that need
to be installed, before even attempting to build anything. This is the approach
taken by Quicklisp. The client downloads a set of files containing this index
from any distributions that it is configured to consult (such as the primary
distribution, also named ``quicklisp'', hosted at
\url{https://beta.quicklisp.org/dist/quicklisp.txt}).

While we have described each of these tasks as happening in distinct stages, in
reality the tasks can be jumbled together. For instance, \verb=ql:quickload=
attempts to resolve and install all dependencies before invoking ASDF
operations. However, it's possible that the Quicklisp distribution's index is
missing some information (for example, if a local project is used that has no
info in the index). Therefore, \verb=ql:quickload= handles conditions that ASDF
signals corresponding to missing dependencies by another round of resolution
and installation.

\subsection{Comparison to Other Languages}

To further ground these concepts, we briefly describe the components
responsible for each of these tasks in the Python and Ruby ecosystems.

In Python \cite{python}, building is mostly performed by the Python interpreter
itself. When a package is \verb|import|ed, it is byte compiled if needed and
then loaded into the interpreter. Pip \cite{pip} is the tool predominately used
for dependency installation and resolution. During resolution, pip uses
metadata gathered from the Python Package Index (PyPI) \cite{pypi} and a
\verb|setup.py| file for packages not in the index.

In Ruby \cite{ruby}, building is performed by a combination of the gem tool
\cite{gem} and the Ruby interpreter. Gem handles building native extensions
(foreign libraries) while the Ruby interpreter loads \verb|require|d packages
much like Python's interpreter. Dependency resolution and installation is
typically handled with a combination of gem and bundler \cite{bundler}. Gem is
largely used for installing releases locally, while bundler is used for project
specific installs. Both gem and bundler can consult indices such as the one
hosted at \url{https://rubygems.org/}.

\section{Features and Philosophy}

CLPM has a set of features that, while commonly found in language specific
package managers for other languages, have not seen wide adoption in the Common
Lisp community. The most important of these include:

\begin{enumerate}
\item the ability to download over HTTPS,
\item the ability to reason over dependency version constraints during
  resolution,
\item the ability to manage both global and project-local contexts,
\item the ability to ``lock'' (or ``pin'', ``freeze'', etc.) dependencies to
  specific versions and replicate that setup exactly on another machine,
\item the ability to compile to a standalone executable with a robust command
  line interface (CLI) for easy interfacing with shell scripts and other
  languages,
\item and the ability to install development releases directly from a project's
  VCS.
\end{enumerate}

While CLPM is not the first to implement most of these features for Common
Lisp, we believe it is the first to do so in a complete package that also
places minimal constraints on the workflow of the developer using it. For
example, the quicklisp-https \cite{quicklisp-https} project adds HTTPS support
to the Quicklisp client, but it does so at the cost of requiring Dexador
\cite{dexador} (and all of its dependencies, including foreign SSL libraries)
are loaded into the development image.

Another example is the qlot \cite{qlot} project. It adds project-local contexts
(but not global), locking, a CLI, and installing directly from VCS. However, it
still requires that the Quicklisp client is installed and loaded. Until very
recently its executable and CLI required Roswell \cite{roswell}. To date its
executable is not distributed and must be built locally.

To our knowledge, no other solution exists that attempts to include version
constraints during resolution. To wit, Quicklisp's index format completely
elides the version constraint in its dependency lists. So no project manager
that uses only Quicklisp style indices for dependency resolution can ensure
that a system's version constraints are satisfied.

As just alluded, the most important guiding principle of CLPM is that it should
place minimal constraints on developer workflow. An example of this principle
in action is illustrated by how CLPM is distributed. For Linux systems, a
static executable is provided that runs on a wide variety of distributions with
no dependencies on foreign libraries (some features, such as VCS integration,
require other programs, such as git, to be installed). For MacOS systems, a
binary is provided that requires only OS provided foreign libraries. And on
Windows, CLPM is distributed using a standalone installer and again depends
only on OS provided foreign libraries.

% More examples of this principle in action are provided in
% Sections~\ref{sec:architecture}~and~\ref{sec:worker-non-portable-code}.

Perhaps the second most important principle is that CLPM should be highly
configurable, yet provide a set of sane and safe defaults. A concrete example
of this is shown by CLPM's behavior when installing or updating projects. By
default, CLPM will describe the change about to be performed and require
explicit user consent before making the modification. However, this behavior
can be changed for developers that like to live on the edge or otherwise have
complete trust in CLPM and the projects they are installing.

\section{Design}

CLPM participates in both the installation and resolution phases of dependency
management. It leaves building completely up to ASDF. In this section we
discuss the design of CLPM and how it completes both of these tasks. First, we
describe the overall architecture, including the three main CLPM
components. Second, we describe some of the benefits the architecture provides
and how CLPM leverages them. Third, we describe where CLPM locates the metadata
needed for dependency resolution. Last, we describe global and project-local
contexts in CLPM.

\subsection{Architecture}\label{sec:architecture}

CLPM is split into two user facing components: the worker and the client, as
well as one internal component: the groveler. The worker is a standalone
executable that is responsible for all the heavy lifting. The worker interacts
with the network to download releases and metadata, performs dependency
resolution, unpacks archives, and manages contexts. The worker is implemented
in Common Lisp and distributed both as a precompiled executable (static
executable for Linux) and source code for those that want to compile it
themselves. It exposes both a CLI interface and a REPL interface. The CLI
interface allows for easy integration with tools such as shell scripts and
continuous integration services. The REPL interface is used primarily by the
CLPM client.

The client is a small system, written in portable Common Lisp, with ASDF/UIOP
as its only dependency. The client is meant to be loaded into a Lisp image to
facilitate the interactive management of dependencies and development of
code. It does this by exposing a set of functions corresponding to the
operations the worker can perform as well as integrating with ASDF's system
search functions. Additionally, it has functionality to remove itself from an
image if it is no longer required (such as when dumping an executable). In
order to interact with the worker, the client spawns a new worker process,
starts its REPL, and they communicate back and forth with S-expressions.

The last component is the groveler. Ideally users never interact directly with
this component. Instead, it is used by the worker to gather the metadata needed
for dependency resolution from systems that are not present in any index. For
example, the groveler is used to extract dependencies from development versions
of projects.

An aspect that makes extracting this metadata difficult is ASDF system
definition files can contain arbitrary code and can require that certain
dependencies be loaded before a system can even be correctly parsed. As such,
it may not be possible to extract this metadata without running arbitrary code.

To address this, the groveler consists of a small set of portable Common Lisp
functions that the worker loads into a fresh Common Lisp instance. Once loaded,
the groveler and worker communicate via S-expressions, with the worker
specifying which systems to load and extract information from and the groveler
reporting back the information as well as any errors. If there is an error, the
worker addresses it by recording any missing dependencies in the dependency
resolution process, loading them into the groveler, and trying again. The
worker keeps track of the systems loaded in the groveler and starts a new
groveler process if needed (e.g., the groveler has v1 of system A loaded, but a
dependency determined later in the resolution process requires v2).

Additionally, CLPM has experimental support to run the groveler in a
sandbox. This sandbox has reduced permissions and cannot write to much of the
file system. This sandbox is currently implemented using Firejail
\cite{firejail} on Linux, but we desire to add support for other OSes and
sandbox solutions.

\subsection{Dependencies and Non-portable Code in the Worker}\label{sec:worker-non-portable-code}

One benefit of the worker/client split is that it enables the worker to freely
leverage both dependencies (Common Lisp and foreign) and non-portable
implementation features.

There are two benefits with respect to dependencies in the worker. First, the
worker can reuse code without worrying about interfering with the code the user
wishes to develop. This interference may manifest itself in many ways,
including package nickname conflicts, version incompatibilities (CLPM needs
version x of a system, but the user's code needs version y), or image size
(e.g., the user's code uses few dependencies and they care about the final size
of the image for application delivery).

Second, the worker can use foreign libraries that it is unlikely the user needs
loaded for their development. A prime example of this is libssl, which CLPM
uses on Linux and MacOS systems to provide HTTP support. A second-order benefit
of this approach is that CLPM can statically link foreign libraries into the
worker so that the user does not even need to have them installed locally (if
they install a pre-compiled version of CLPM at least).

The ability for the worker to freely use dependencies has an additional
knock-on effect: the development of CLPM can help improve the state of Common
Lisp libraries at large. To date, the development of CLPM has resulted in at
least ten issues and merge requests being reported to upstream maintainers,
eight of which have been merged or otherwise addressed. Additionally, it has
been the motivating factor behind several contributions to SBCL to improve musl
libc support.

The worker routinely needs to perform operations that require functionality
beyond the Common Lisp specification. For example: loading foreign libraries,
interfacing with the OS to set file time stamps when unpacking archives, and
network communication to download code. While portability libraries exist for
many of these features, they are not perfect and do not necessarily extend
support to all implementations \cite{portability}. The worker/client split
allows us to choose a small number of target implementations and focus our
testing and distribution efforts using only those implementations while not
worrying about restricting what implementation the user uses to develop their
code.

Currently, CLPM targets only SBCL \cite{sbcl} for worker support. This is due
to SBCL's broad compatibility with OSes and CPU architectures. We look to
extend worker support to at least one other implementation before CLPM reaches
v1.0. The next implementation targeted by the worker will likely be ABCL
\cite{abcl} due to the ubiquity of the Java Virtual Machine.

\subsection{Indices}

Project indices are used to advertise projects and metadata about those
projects, such as their released versions, what systems are available in each
release, and the version number and dependencies of those systems. The most
widely used project index in the Common Lisp community is the primary Quicklisp
distribution. CLPM is tested against this index and has full support for
interacting with it.

While there are other Quicklisp-like project indexes in the wild that work with
the Quicklisp client, such as Ultralisp \cite{ultralisp}, CLPM may not work
with all of them. This incompatibility is largely due to a lack of formal
specification as to what constitutes a Quicklisp distribution, including what
files are required and their contents. For instance, at the time of writing
CLPM does not work with the Ultralisp distribution because Ultralisp does not
publish a list of all its historical versions, unlike the primary Quicklisp
distribution (see:
\url{https://beta.quicklisp.org/dist/quicklisp-versions.txt}).

As discussed above, the Quicklisp distributions strip the version constraints
from dependencies. However, CLPM actually supports reasoning over those
constraints. This was the main motivation to develop and support Common Lisp
Project Indices (CLPI). CLPI is part of the overarching CLPM project and seeks
to fully document an index format as well as add features that CLPM can take
advantage of that are missing from Quicklisp distributions. CLPI adds fields to
record both a system's version (Quicklisp indices provide only the date at
which the project snapshot was taken) and the version constraints of its
dependencies.

A second major difference is that projects and systems are the top-level
concepts in CLPI instead of distribution version. Additionally, CLPI defines
the majority of its files to be append-only. These properties allow CLPM to be
more efficient in terms of network usage as only the metadata for projects
potentially needed in the context are transferred and metadata on new releases
can be fetched incrementally over time.

\subsection{Contexts}

CLPM manages both global and project-local contexts. A context is defined as a
set of project indices in use, a set of constraints describing the projects
(and their versions) that should be available in the context, and a set of
project releases that satisfy both the explicit constraints and the implicit
constraints added by every project in the context (i.e., transitive
dependencies).

Each global context is named by a string. CLPM provides tools that generate
ASDF source registry configuration files so that you can add these global
contexts to your default registry and have access to all projects installed in
them without the need for CLPM at all (after installation, at least).

A project-local context (also known as a {\it bundle}) is named by a pathname
that points to a file containing the first two elements of a context (the
indices and constraints). After this context is installed, all of the context
information is located in a file next to the file that names the context. The
names of these files are typically {\tt clpmfile} and {\tt clpmfile.lock}. Both
of these files are designed to be checked into a project's source control and
contain enough information to reproduce the context on another machine.

Both CLPM's CLI and the client provide commands to add new constraints to a
context, update an entire context so that every project is at the latest
release that satisfies all constraints, or update just a subset of the projects
in the context. Additionally, the CLI provides commands that can execute other,
arbitrary, commands in an environment where ASDF is configured to find only the
projects installed in the desired context.

Global contexts are largely inspired by Python virtual environments created by
the virtualenvwrapper project \cite{virtualenvwrapper}. Project-local contexts
are largely inspired by Ruby's Bundler project \cite{bundler}.

\section{Use}

In this section we provide some examples of how to use CLPM. We focus on
project-local contexts (bundles) as we believe these are a more broadly useful
feature than global contexts.

\subsection{Installing CLPM}

Tarballs (and MSI installers) of the most recent CLPM release, along with up to
date documentation, can be found at \url{https://www.clpm.dev}. Windows users
merely need to run the installer. Linux and MacOS users need to only unpack the
tarball in an appropriate location (typically {\tt /usr/local/}). If you wish
to install from source, the CLPM source code (as well as its issue tracker) can
be found on the Common Lisp Foundation Gitlab instance at
\url{https://gitlab.common-lisp.net/clpm/clpm}.

After installing CLPM, it is recommended that you configure ASDF to find the
CLPM client. To do this, simply run {\tt clpm client source-registry.d} and
follow the instructions printed to the screen.

Last, you may wish to consider loading the CLPM client every time you start
your Lisp implementation. You can determine the currently recommended way of
doing so by running {\tt clpm client rc} and reading the printed
instructions. The remainder of this section assumes you have CLPM installed and
have a REPL where the CLPM client is loaded.

\subsection{Configuration}

CLPM reads its configuration files from the \verb|~/.config/clpm/| folder on
non-Windows OSes and \verb|%LOCALAPPDATA%\clpm\config\| on
Windows. The file {\tt clpm.conf} contains most of the configuration, but the
defaults should be sufficient for this demo.

The file {\tt sources.conf} contains a list of sources for CLPM to use when
finding projects and their metadata. For this demo, it will be easiest if we
populate this file with a pointer to the primary Quicklisp distribution. Place
{\it only one} of the following forms in {\tt sources.conf}.

This form uses the primary Quicklisp distribution directly. Note that we are
fetching it over HTTPS. All of the primary Quicklisp distribution's files are
served over both HTTPS and HTTP, even though the Quicklisp client itself can
only use HTTP.

\begin{verbatim}
(:source "quicklisp"
 :url "https://beta.quicklisp.org/dist/quicklisp.txt"
 :type :quicklisp)
\end{verbatim}

This form uses a mirror of the primary Quicklisp distribution. This mirror
exposes the same data, formatted using the CLPI specification. Using this
source will result in CLPM performing some operations faster (because it can
download only the needed data). However, this source may take some time to be
updated after a new version of the Quicklisp distribution is released.

\begin{verbatim}
("quicklisp"
 :url
 "https://quicklisp.common-lisp-project-index.org/"
 :type :ql-clpi)
\end{verbatim}

\subsection{Downloading the Demo System}

We have a simple project designed for use in CLPM demos. It is located at
\url{https://gitlab.common-lisp.net/clpm/clpm-demo}. Clone it anywhere on your
file system and checkout the {\tt els-21} branch.

\subsection{Creating a Bundle}

We will now create a bundle for the demo system. The simplest possible bundle
is one that uses a single project index and specifies that all constraints come
from .asd files. To create such a bundle you need to perform two
actions. First, a {\tt clpmfile} must be created. This can be done at the REPL
by evaluating:

\begin{verbatim}
(clpm-client:bundle-init
 #p"/path/to/clpm-demo/clpmfile"
 :asds '("clpm-demo.asd" "clpm-demo-test.asd"))
\end{verbatim}


After evaluating this, you will have a {\tt clpmfile} that looks similar to the
following:

\begin{verbatim}
(:api-version "0.3")

(:source "quicklisp"
 :url "https://beta.quicklisp.org/dist/quicklisp.txt"
 :type :quicklisp)

(:asd "clpm-demo.asd")
(:asd "clpm-demo-test.asd")
\end{verbatim}

Notice that the first statement in the {\tt clpmfile} declares the bundle API
in use. This allows for the file format to evolve over time while maintaining
backward compatibility. Second, notice that every directive is simply a
plist. This makes it trivial for any Common Lisp project to read and manipulate
the file.

After the {\tt clpmfile} is created, the bundle dependencies must be resolved
and installed. To do this, evaluate the following at the REPL:

\begin{verbatim}
(clpm-client:install
 :context #p"/path/to/clpm-demo/clpmfile")
\end{verbatim}

CLPM does not believe in modifying a context without explicit permission from
the developer. As such, the client will by default produce a condition before
it performs any modifications. This condition describes the actions that are
about to be performed and has two restarts established for it: one to approve
and perform the changes, and one to abort. Therefore, upon evaluating the above
form, you will be dropped into the debugger to approve the changes. Of course,
this behavior can be customized.

This will create a file {\tt clpmfile.lock}. This file should not be modified
by hand. It contains all the information necessary for CLPM to reproduce the
bundle on another machine, including a complete dependency graph.

\subsection{Activating the Bundle}

Once a bundle has been installed, the next step is to activate it. The
activation process configures ASDF so that it can locate all systems in the
bundle. The activation process also optionally integrates the CLPM client with
ASDF. This allows CLPM to notice when you try to find a system that is not
present in the bundle and can provide options on adding it to the bundle. This
integration is enabled by default and discussed more next. You can activate the
bundle by evaluating:

\begin{verbatim}
(clpm-client:activate-context
 #p"/path/to/clpm-demo/clpmfile")
\end{verbatim}

The CLPM CLI has an equivalent command that configures ASDF using only
environment variables. This feature is particularly useful when running scripts
or when running tests via CLI. For example, to run an SBCL process where ASDF
is configured to use only the systems in the bundle {\it without} also needing
to have the client loaded, simply run {\tt clpm bundle exec -{}- sbcl} in the
same directory as the {\tt clpmfile}.

You can now load the demo system by evaluating:
\begin{verbatim}
(asdf:load-system "clpm-demo")
\end{verbatim}

\subsection{Modifying the Bundle}

The most common modification made to a bundle is to add more dependencies to
it. If your system acquires a new dependency you have two options on how to add
it to the bundle. The first option is to explicitly reinstall the bundle using
{\tt install}. This will find any new dependency and install it, while trying
its best to not change the installed versions of the projects already in the
bundle.

The second option is to just use {\tt asdf:load-system} to reload your
system. If you have the client's ASDF integration enabled, the client will
notice that the system is missing from the bundle and take action. The default
action is to signal a condition informing the developer of the situation with
several restarts in place. Following CLPM's guiding principles, however, this
behavior can be modified to, for example, automatically install the dependency
\`a la the Quicklisp client.

To see this in action, open the file {\tt clpm-demo-test.asd} and add
\verb|"fiveam"| to the \verb|:depends-on| list. Then try to run the system's
test suite by evaluating \verb|(asdf:test-system "clpm-demo")|. Fiveam was
originally not part of the bundle so ASDF signals a condition when it tries to
load it. However, we can use the
\verb|reresolve-requirements-and-reload-config| restart to install fiveam and
then see the tests pass.

\subsection{Using Development Versions}

Now, let's say that fiveam's author has added a really cool new feature that
you want to start using before it's released into Quicklisp. You can easily do
this by telling CLPM to install fiveam directly from git. Simply add the
following directive to your \verb|clpmfile|.

\begin{verbatim}
(:github "fiveam"
 :path "lispci/fiveam"
 :branch "master")
\end{verbatim}

Then evaluate \verb|(clpm-client:install)|. Note that you do not have to
specify the context if you have already activated it. If you evaluate
\verb|(asdf:test-system "clpm-demo")| again, you can see that it loads fiveam
from a different path than it did previously.

\subsection{Developing a Dependency}

Last, let's say that you would like to add a new feature to fiveam and use it
in clpm-demo. You should not modify any files installed by CLPM, as they may be
reused between projects and the \verb|.git| folder is stripped from any
dependencies installed directly from their git repo.

Instead, you should clone the project you want to work on yourself and tell
CLPM to use that checkout instead of the one it installs. To demonstrate this,
clone fiveam (\url{https://github.com/lispci/fiveam}) so that it is next to the
clpm-demo repository. Then, create the file \verb|.clpm/bundle.conf| inside the
clpm-demo repository with the following contents:

\begin{verbatim}
(version "0.2")

((:bundle :local "fiveam")
 "../fiveam/")
\end{verbatim}

This tells CLPM to not install fiveam (as you have already done it) and to use
the git repository located at \verb|../fiveam/| (relative to clpm-demo's
\verb|clpmfile|) when resolving dependencies.

A \verb|clpm-client:hack| function that performs these steps automatically is
currently being developed. It is slated for inclusion in CLPM v0.5.

\section{Conclusion and Future Work}

We have presented CLPM --- the Common Lisp Project Manager. CLPM introduces
many features to the Common Lisp community that are taken for granted in other
programming language specific package managers. Key among these features are
HTTPS support, a standalone CLI to the worker, global and project-local context
management, and lock files. Additionally, CLPM adds these features without
forcing a particular development practice or environment and without
sacrificing ``Lispy-ness'' or interactive development.

We plan to continue developing CLPM and continue adding useful features. Two
planned features of note are an extensible architecture for client/worker
communication and the ability to add scripts from an installed project to a
user's {\tt PATH}. The former would enable a myriad of new configurations,
including CLPM workers deployed as persistent daemons that communicate with
clients over network sockets or setups that are Dockerized or otherwise
isolated from other processes on the system.

In addition to improvements to CLPM itself, we aim to continue contributing
back to the upstreams of our dependencies. Notably, we are in the process of
interacting with the ASDF developers to add support for more expressive version
numbers and dependency version constraints in {\tt defsystem} forms.

Last, it is our strong preference to enable developers to use the complete
power of CLPM without introducing a split in the community with regards to
project indices. Therefore, we would like to take ideas and lessons learned
from CLPI and integrate them into Quicklisp distributions.

\bibliographystyle{plainnat}
\bibliography{main}

\end{document}